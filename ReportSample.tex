\documentclass[runningheads]{llncs}
%---- Coding----%
\usepackage[utf8]{inputenc}  % for Unix and Windows
\usepackage[T1]{fontenc}
\usepackage{graphicx}
\usepackage{url}
\usepackage{llncsdoc}
%----- Math symbols ---%
\usepackage{amsmath}
\usepackage{amssymb}
\usepackage{enumerate}

\begin{document}

\mainmatter
\title{Deep Learning Industrial Challenge (DLIC 2017) 
}
\titlerunning{Title}
\author{Akbar Shadakov}
\authorrunning{Your Name}
\institute{SPOTSeven Lab\\
 Cologne University of Applied Sciences}
\date{May 2014}
\maketitle

\begin{abstract} abstract comes here
\end{abstract}

\section{Introduction}
This article illustrates how an existing algorithm, namely simulated annealing, can be tuned using the
SPOT framework.

Related work.

Section~\ref{sec:tsp} introduces TSP.
Section~\ref{sec:sann} describes Simulated Annealing.

\section{Traveling Salesman Problems}\label{sec:tsp}
\subsection{Definitions}
\subsection{Implementation in R}

\section{Simulated Annealing}\label{sec:sann}
\subsection{The Algorithm}

\subsection{Implementation in R}

\section{Sequential Parameter Optimization}
\subsection{Overview}
 The SPOT package can be installed from within R using the 
\begin{verbatim}
install.packages("SPOT")
\end{verbatim}
command. Alternatively, SPOT can 
downloaded from the
comprehensive R  archive network at \url{http://CRAN.R-project.org/package=SPOT}.
The latter procedure is recommended for the experienced R user only. 
SPOT is one possible implementation of the \emph{sequential parameter optimization}\/
(SPO) framework introduced in~\cite{Bart06a}.
For a detailed documentation of the functions from the SPOT package, the
reader is referred to the package help manuals.
\cite{Bart12i} introduces the SPOT and applications.
\subsection{Interfacing With Simulated Annealing}
In Figure~\ref{fig:tune2} the tuning is shown.


%\begin{figure}
%\includegraphics[width=0.9\linewidth]{cnt0015Ackley83.pdf}
%\caption{Tuning with SPOT}
%\label{fig:tune2}
%\end{figure}

\section{Experiments}

\section{Results}

\section{Discussion}

\section{Summary}
Knuth says:
\cite{knuth2005art}

\bibliographystyle{splncs03}
\bibliography{ReportSample}

\end{document}